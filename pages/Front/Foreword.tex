\phantomsection
\addcontentsline{toc}{chapter}{Foreword}
\fancyhead[C]{{\LARGE Foreword}}
\noindent
\lett{T}{he} publication of this edition of the Book of Common Prayer, carefully edited and directed toward those in the Antiochian Western Rite Vicariate, marks a contemplative moment of continued growth of the Orthodox Catholic faith in North America and in the territories of British Christendom. Much has already been written about the nearly universal acceptance of the Western Rite among our Holy Fathers in modern times, and so this foreword is not concerned with such a topic. The message I hope to leave with the faithful is the importance of beauty, and the importance of Orthodox Christianity being the Great Healer of nations.\\

The Orthodox Church has always emphasized the importance of beauty in the salvation of mankind, for as it says in the Epistle of James,

\begin{quoting}\noindent
	Every good gift and every perfect gift is from above, and cometh down from the Father of lights, with whom is no variableness, neither shadow of turning. ---James~1:17
\end{quoting}

In our homeland of North America, many of us have been led to Orthodoxy by the beauty of the English tradition of prayer which was given to our forefathers (according to the flesh) by way of our English and Scottish heritage. As the peoples of the British Isles arrived in the colonial New World, so too did their prayer tradition which was rooted in a deep connection to the blessed Dowry of Mary, that is, the Albion of Saint Alfred and the Three Holy Patrons: St. Edmund, St. Edward, and St. Gregory the Great. Even the Congregationalists, Puritans, and ``Low Churchmen'' could not avoid retaining an echo of the ancient Liturgy within their houses of prayer. \\

This deep connection to the England once known as Mary's Dowry is what allowed for beauty to thrive in a deeply desolate new world. Amidst the darkness of isolation and uncertainty, New Englanders and Southern Americans built lasting houses of prayer adorned with the beauty of Christ's Name and images of His saints. Within those halls, the men and women of an uncertain new world prayed the collects, hymns, and prayers of an ancient inheritance, albeit commingled with new and unfamiliar doctrines and liturgics that set a broad path against what was once dedicated to the very Narrow Way of Christ. In modern times, the Christians of the English tradition have found themselves near shipwreck and in danger of losing the beauty of their rich past. Though the valuable buildings, real-estate, and stained glass have passed on to ``the Religion of the Future,'' as our holy father St. Seraphim of Platina called it, the prayers and liturgical life of English Christians can be still be made firm in the life giving Water of Christ. This life giving water runs out from the side of the Living Christ, from His Church, which boldly proclaims the Catholic faith without shame.\\

The thirst for the living water of the ancient Orthodoxy of the Christian life was never fully stomped out by the errors of the dialectical tension of Roman Catholicism and Anglicanism. Much has been said before about the elegant writing of William Laud, Richard Hooker, Lancelot Andrewes, and George Herbert. And so I wish to offer a sometimes forgotten example from the Right Reverend William Seabury of Connecticut. Rt. Rev. Seabury was the first Angli-can bishop consecrated in the United States and, writing in favor of aligning the American Anglican ethos with that of the Scottish prayerbook, he wrote:

\begin{quoting}\noindent
	The grand fault in that [1662] office [of Holy Communion] is the deficiency of a more formal oblation of the elements, and of the invocation of the Holy Ghost to sanctify and bless them. The Consecration is made to consist merely in the Priest's laying his hands on the elements and pronouncing ``This is my body,'' etc., which words are not consecration at all, nor were they addressed by Christ to the Father, but were declarative to the Apostles . . . The efficacy of Baptism, of Confirmation, of Orders, is ascribed to the Holy Ghost, and His energy is implored for that purpose; and why he should not be invoked in the consecration of the Eucharist, especially as all the old Liturgies are full to the point, I cannot conceive. ---\emph{The Life and Correspondence of Samuel Seabury}, p. 354
\end{quoting}

We see in this early American bishop a desire to return to the ancient liturgies, to live within the uncreated energies experienced and preached by St. Paul, and to find what had been lost after the Great Schism and the deeply regrettable brother wars of the Reformation which led to the destruction, desolation, and murder of European Christianity.\\

Herein we find the purpose of the 2025 Proposed Book of Common Prayer: to bring forth the Good, to let go of all that which is contrary to the Truth, and to heal what is infirm. Following carefully the recommendations of the fathers of the Antiochian and Russian Orthodox Churches, the 2025 Proposed Book of Common Prayer has only sought to remain loyal to what the Orthodox Church has already observed and ruled. Recalling the conclusion to the Russian Observations on the American Prayerbook, it is worth restating their findings:

\begin{quoting}\noindent
	The committee, after reviewing these ``Observations,'' allowed in general the possibility that if Orthodox parishes, composed of former Anglicans, were organized in America, they might be allowed, at their desire, to perform their worship according to the ``Book of Common Prayer,'' but only on condition that the following corrections were made in the spirit of the Orthodox Church. ---\emph{Russian Observations upon the American Prayer Book} 2:VI
\end{quoting}

The 2025 Proposed Book of Common Prayer is not a restoration project, nor is it a historical recreation. Rather it is a spiritual surgery for those of the Anglican patrimony who seek communion with the One, Holy, Catholic, and Apostolic Church and to retain that which the ancient Church provided, influenced, and now attempts to sustain.\\

Continued prayers are also asked for all those of the Western Rite, so that a spirit of pretest, pride, or confusion may never yoke itself to the Latin and English presence within the Orthodox world. All things are submitted to the Holy Church so that Christ may judge, through His living Mystical Body.\\

\rightline{\emph{The Blessings of Our Lord Jesus Christ,}}

\vspace{1\baselineskip}

\rightline{The Reverend Father Justin Slaughter Doty}

\rightline{\emph{Priest, St. Euphemia Orthodox Christian Church}}

\rightline{\emph{Overseeing St. Botolph Orthodox Chapel}}
