%Dissatisfaction with the 1928 BCP's Rite of Matrimony (and the American tradition altogether) is readily apparent in the various western, and especially American, communities. Therefore, the 1549 Rite is provided. It is so beautiful and magnificent, it needs no change in order to be used in the Orthodox Church, but due to issues of prudence, very light revisions are made and noted.
\section{Sacrament of Matrimony}
\fancyhead[RO,LE]{\textit{Matrimony}}
\fancyhead[RE,LO]{}
\begin{secrubric}
First the banns must be posted for three Sundays or Holy Days. If the persons that would be married dwell in divers parishes, the banns must be posted in both parishes, and the Curate of the one parish shall not solemnise matrimony betwixt them, without a certificate of the banns being thrice asked from the Curate of the other parish.\par
At the day appointed for Solemnisation of Matrimony, the persons to be married shall come into the body of the church, with their friends and neighbours. And there the priest shall thus say,
\end{secrubric}
\lett{D}{early} beloved friends, we are gathered together here in the sight of God, and in the face of his congregation, to join together this man and this woman in holy matrimony, which is an honourable estate instituted of God in paradise, in the time of man's innocency, signifying unto us the mystical union that is betwixt Christ and his Church: which holy estate, Christ adorned and beautified with his presence, and first miracle that he wrought in Cana of Galilee, and is commended by Saint Paul to be honourable among all men; and therefore is not to be enterprised, nor taken in hand unadvisedly, lightly, or wantonly, to satisfy men's carnal lusts and appetites, like brute beasts that have no understanding: but reverently, discretely, advisedly, soberly, and in the fear of God. Duly considering the causes for the which matrimony was ordained.
\par
\subsection{Exhortation}
%Original: One cause was
%The change is made to ensure that procreation is not seen as merely as `one cause' but the primary cause of marriage.
First it was ordained for the procreation of children, to be brought up in the fear and nurture of the Lord, and praise of God.\par
Secondly it was ordained for a remedy against sin, and to avoid fornication, that such persons as be married, might live chastely in matrimony, and keep themselves undefiled members of Christ's body.\par
Thirdly for the mutual society, help, and comfort, that the one ought to have of the other, both in prosperity and adversity. Into the which holy estate these two persons present: come now to be joined. Therefore if any man can shew any just cause why they may not lawfully be joined so together: Let him now speak, or else hereafter for ever hold his peace.
\begin{rubric}
	And also speaking to the persons that shall be married, he shall say,
\end{rubric}
\lett{I}{require} and charge you (as you will answer at the dreadful day of judgement, when the secrets of all hearts shall be disclosed) that if either of you do know any impediment, why ye may not be lawfully joined together in matrimony, that ye confess it. For be ye well assured, that so many as be coupled together otherwise than God's Word doth allow: are not joined of God, neither is their matrimony lawful.
\begin{rubric}
    At which day of Marriage, if any man do allege and declare any impediment, why they may not be coupled together in Matrimony, by God's law, or the laws of this Realm; and will be bound, and sufficient sureties with him, to the parties; or else put in a caution (to the full value of such charges as the persons to be married do thereby sustain) to prove his allegation: then the solemnisation must be deferred, until such time as the truth be tried.
\end{rubric}

\subsection{Vows}
\begin{rubric}
    If no impediment be alleged, then shall the Curate say unto the man,
\end{rubric}
\lett{N}{., wilt} thou have this woman to thy wedded wife, to live together after God's ordinance in the holy estate of matrimony? Wilt thou love her, comfort her, honour, and keep her in sickness and in health? And forsaking all other keep thee only to her, so long as you both shall live?\par
℟. I will.\par
\begin{rubric}
	Then shall the Curate say unto the woman,
\end{rubric}
\lett{N}{., wilt} thou have this man to thy wedded husband, to live together after God's ordinance, in the holy estate of matrimony? Wilt thou obey him, and serve him, love, honour, and keep him in sickness and in health? And forsaking all other keep thee only to him, so long as you both shall live?\par
℟. I will.\par
\begin{rubric}
	Then shall the Minister say,
\end{rubric}\noindent
Who giveth this woman to be married to this man?
\begin{rubric}
	And the Minister receiving the woman at her father's or friend's hands: shall cause the man to take the woman by the right hand, and so to give their troth to each other: The man first saying,
\end{rubric}
\lett{I}{N.} take thee \textit{N.} to my wedded wife, to have and to hold from this day forward, for better, for worse, for richer, for poorer, in sickness, and in health, to love and to cherish, 
%Should this be changed to the 1662 wording?
til death us depart: according to God's holy ordinance: And thereto I plight thee my troth.
\begin{rubric}
	Then shall they loose their hands, and the woman taking again the man by the right hand shall say,
\end{rubric}
\lett{I}{N.} take thee \textit{N.} to my wedded husband, to have and to hold from this day forward, for better, for worse, for richer, for poorer, in sickness, and in health, to love, cherish, and to obey, till death us depart: according to God's holy ordinance: And thereto I give thee my troth.

\begin{rubric}
	Then shall they again loose their hands, and the man shall give unto the woman a ring, and 
%Since the gifting of gold is not in common usage, this is made optional.
	optionally other tokens of spousage, as gold or silver, laying the same upon the book: And the Priest taking the ring shall deliver it unto the man: to put it upon the fourth finger of the woman's left hand. And the man taught by the Priest, shall say,
\end{rubric}\par\noindent
With this ring I thee wed: (This gold and silver I thee give:) with my body I thee worship: and with all my worldly goods I thee endow. In the name of the Father, and of the Son, and of the Holy Ghost. Amen.
%It is notable that the rite itself does not include a place for the woman to put a ring on the man's finger. While the contemporary practice is not itself impious, the tradition rite is excellent in its expression of the Christian doctrine of marriage and ought to be retained and this rite not modified. Of course, after the ceremony, as an extra-liturgical devotion, the woman ought to place a ring upon the left hand of the man to avoid impropriety in twenty-first century America.

\subsection{Solemnisation of Matrimony}
\begin{rubric}
	Then the man leaving the ring upon the fourth finger of the woman's left hand, the Minister shall say,
\end{rubric}
\letuspray
\lett{O}{eternal} God, Creator and preserver of all mankind, giver of all spiritual grace, the author of everlasting life: Send thy blessing upon these thy servants, this man, and this woman, whom we bless in thy name, that as Isaac and Rebecca (after bracelets and Jewels of gold given of the one to the other for tokens of their matrimony) lived faithfully together; So these persons may surely perform and keep the vow and covenant betwixt them made, whereof this ring given, and received, is a token and pledge. And may ever remain in perfect love and peace together; And live according to thy laws; through Jesus Christ our Lord. \textit{Amen.}
\begin{rubric}
	Then shall the Priest join their right hands together, and say,
\end{rubric}
\begin{center}
    \textsc{Those whom God hath joined together:\\let no man put asunder.}
\end{center}
\begin{rubric}
	Then shall the Priest speak unto the people.
\end{rubric}
\lett{F}{orasmuch} as \textit{N.} and \textit{N.} have consented together in holy wedlock, and have witnessed the same here before God and this company; And thereto have given and pledged their troth either to other, and have declared the same by giving and receiving gold and silver, and by joining of hands: I pronounce that they be man and wife together. In the name of the Father, of the Son, and of the Holy Ghost. Amen.
\begin{rubric}
	And the Minister shall add this Blessing.
\end{rubric}
\lett{G}{od} the Father bless you {\ding{64}} God the Son keep you: God the Holy Ghost lighten your understanding: The Lord mercifully with his favour look upon you, and so fill you with all spiritual benediction, and grace, that you may have remission of your sins in this life, and in the world to come life everlasting. \textit{Amen.}
\subsection{Thanksgiving}
\begin{rubric}
	Then shall they go into the choir, and the Ministers or Clerks shall say or sing, this psalm following.
\end{rubric}
\subsubsection{Psalm 128. \textit{Beati omnes}}
\lett{B}{lessed} are all they that fear the Lord : and walk in his ways.\par
\secondline{\psanum{2}For thou shalt eat the labours of thine hands : O well is thee, and happy shalt thou be.}
\thirdline{\psanum{3}Thy wife shall be as the fruitful vine : upon the walls of thine house.}
\psanum{4}Thy children like the olive-branches : round about thy table.\par
\psanum{5}Lo, thus shall the man be blessed : that feareth the Lord.\par
\psanum{6}The Lord from out of Sion shall so bless thee : that thou shalt see Jerusalem in prosperity all thy life long.\par
\psanum{7}Yea, that thou shalt see thy children's children : and peace upon Israel.\par
     ℣. Glory be to the Father and to the Son and to the Holy Ghost.\par
    ℟. As it was in the beginning, is now, and ever shall be, world without end. Amen.
    
\begin{inhead}
    or,
\end{inhead}

\subsubsection{Psalm 67. \textit{Deus misereatur}}
\lett{G}{od} be merciful unto us, and bless us : and shew us the light of his countenance, and be merciful unto us;\par
\secondline{\psanum{2}That thy way may be known upon earth : thy saving health among all nations.}
\thirdline{\psanum{3}Let the people praise thee, O God : yea, let all the people praise thee.}
\psanum{4}O let the nations rejoice and be glad : for thou shalt judge the folk righteously, and govern the nations upon earth.\par
\psanum{5}Let the people praise thee, O God : let all the people praise thee.\par
\psanum{6}Then shall the earth bring forth her increase : and God, even our own God, shall give us his blessing.\par
\psanum{7}God shall bless us : and all the ends of the world shall fear him.\par
    ℣. Glory be to the Father and to the Son and to the Holy Ghost.\par
    ℟. As it was in the beginning, is now, and ever shall be, world without end. Amen.

%MANUAL ADJUSTMENT:
\clearpage
\begin{rubric}
	The psalm ended, and the man and woman kneeling before the altar: the Priest standing toward the altar, and turning his face toward them, shall say,
\end{rubric}
℣. Lord have mercy upon us.\par
℟. Christ have mercy upon us.\par
℣. Lord have mercy upon us.\par
℣. Our Father, who art in heaven, Hallowed be thy Name. Thy kingdom come. Thy will be done, On earth as it is in heaven. Give us this day our daily bread. And forgive us our trespasses, As we forgive those who trespass against us. And lead us not into temptation,\par
℟. But deliver us from evil. Amen.\par
℣. O Lord save thy servant, and thy handmaiden.\par
℟. Which put their trust in thee.\par
℣. O Lord send them help from thy holy place.\par
℟. And evermore defend them.\par
℣. Be unto them a tower of strength.\par
℟. From the face of their enemy.\par
℣. O Lord, hear my prayer.\par
℟. And let my cry come unto thee.\par
\letuspray
\lett{O}{God} of Abraham, God of Isaac, God of Jacob, bless these thy servants, and sow the seed of eternal life in their minds, that whatsoever in thy holy Word they shall profitable learn: they may in deed fulfil the same. Look, O Lord, mercifully upon them from heaven, and bless them: And as thou didst send thy Angel Raphael to Tobit, and Sara, the daughter of Raguel, to their great comfort; so vouchsafe to send thy blessing upon these thy servants, that they obeying thy will, and alway being in safety under thy protection: may abide in thy love unto their lives' end: through Jesu Christ our Lord. \textit{Amen.}

%Rubric from the Sarum Rite provided to distinguish first and second+ marriages.
\begin{rubric}
	It is to be noted that the prayers \blackrubric{O eternal God} and \blackrubric{God the Father bless you} is not to be said in second or third marriages. For neither the man nor the woman, entering into a second union, ought to be blessed again by the Priest; forasmuch as they were blessed on a former occasion, their blessing is not to be repeated. For blessed flesh draweth unto itself flesh that is not blessed.\par
	Blessed Ambrose testifieth to this, saying: `The first marriage is instituted by the Lord; the second is permitted. The first is celebrated with every blessing; the second is without any blessing.'
\end{rubric}
\begin{rubric}
	While there are various blessings given in the solemnisation of matrimony, the prayers specified here, as well as `O merciful Lord' and `O God, which by thy' (as specified in the Votive Mass), are to be omitted. For it is unfitting for a blessing to speak of the unity of Christ and the Church, which is figured in the first marriage, but not in the second. As the Apostle saith to the Corinthians: `They two shall be one flesh.' But he who cleaveth to more than one dissolveth the unity or the covenant of unity. Therefore, that blessing which speaketh of such unity is not to be said in second marriages. And this is true whether the man be digamous or the woman a widow: for blessed flesh draweth unto itself flesh that is not blessed.
\end{rubric}
\begin{rubric}
	Although the second marriage, in itself considered, is a true and perfect sacrament, yet when considered in relation to the first, it hath some defect as a sacrament. For it doth not have the fullness of signification, inasmuch as it no longer represents `one flesh,' as doth the marriage betwixt Christ and the Church. For this reason the blessing is withheld from second marriages.\par
	This is to be understood particularly when both parties are entering into second marriage---that is, second on the part of both the man and the woman. For if a virgin be joined to a man who had before another wife, the nuptial blessing may nonetheless be given, for something of the signification is retained in regard to the first nuptials. Even as a bishop, though he have but one bride (the Church), hath yet many souls espoused within her; but the soul itself may not be the spouse of any other than Christ. To do so were to commit spiritual fornication with the devil, and such union were no true marriage. Wherefore, when a woman marrieth again, the nuptial blessing is not given, by reason of the defect in the sacramental sign.
\end{rubric}
%These prayers are omitted here in order to be placed in the Votive Mass:

% \begin{rubric}
% 	This prayer following shall be omitted where the woman is past childbirth.
% \end{rubric}

%The following blessing is moved into the Votive Mass.
%\begin{rubric}
%This is changed in order to make clear that it is the blessing which effects the bond of matrimony.
%	Then shall the priest bless, ratify, and solemnise the bond of matrimony through this benediction of the man and the woman, saying,
%\end{rubric}

\subsection{Sermon}
\begin{rubric}
	Then shall Holy Communion begin (p. \pageref{MatrimonyMass}). Then be said after the Creed a sermon, wherein ordinarily (so oft as there is any marriage) the office of man and wife shall be declared according to Holy Scripture. Or if there be no sermon, the Minister shall read this that followeth.
\end{rubric}
\lett{A}{ll} ye which be married, or which intend to take the holy estate of matrimony upon you: hear what Holy Scripture doth say, as touching the duty of husbands toward their wives, and wives toward their husbands.\par
    Saint Paul (in his epistle to the Ephesians the fifth chapter) doth give this commandment to all married men.\par
    Ye husbands love your wives, even as Christ loved the Church, and hath given himself for it, to sanctify it, purging it in the fountain of water, through the word, that he might make it unto himself, a glorious congregation, not having spot or wrinkle, or any such thing; but that it should be holy and blameless. So men are bound to love their own wives as their own bodies: he that loveth his own wife, loveth himself. For never did any man hate his own flesh, but nourisheth and cherisheth it, even as the Lord doth the congregation, for we are members of his body, of his flesh, and of his bones. For this cause shall a man leave father and mother, and shall be joined unto his wife, and they two shall be one flesh. This mystery is great, but I speak of Christ and of the congregation. Nevertheless let every one of you so love his own wife, even as himself.\par
    Likewise the same Saint Paul (writing to the Colossians) speaketh thus to all men that be married: Ye men, love your wives and be not bitter unto them.\par
    Hear also what Saint Peter the apostle of Christ, (which was himself a married man,) sayeth unto all men that are married.\par
    Ye husbands, dwell with your wives according to knowledge: Giving honour unto the wife, as unto the weaker vessel, and as heirs together of the grace of life, so that your prayers be not hindered.\par
    Hitherto ye have heard the duty of the husband toward the wife.\par
    Now likewise, ye wives, hear and learn your duty toward your husbands, even as it is plainly set forth in holy scripture.\par
    Saint Paul (in the forenamed epistle to the Ephesians) teacheth you thus:\par
    Ye women submit yourselves unto your own husbands as unto the Lord: for the husband is the wife's head, even as Christ is the head of the Church: And he also is the Saviour of the whole body. Therefore as the Church, or congregation, is subject unto Christ: So likewise let the wives also be in subjection unto their own husbands in all things.\par
    And again he sayeth: Let the wife reverence her husband. And (in his epistle to the Colossians) Saint Paul giveth you this short lesson. Ye wives, submit yourselves unto your own husbands, as it is convenient in the Lord.\par
    Saint Peter also doth instruct you very godly, thus saying, Let wives be subject to their own husbands, so that if any obey not the Word, they may be won without the Word, by the conversation of the wives; While they behold your chaste conversation, coupled with fear, whose apparel let it not be outward, with plaited hair, and trimming about with gold, either in putting on of gorgeous apparel: But let the hid man which is in the heart, be without all corruption, so that the spirit be mild and quiet, which is a precious thing in the sight of God. For after this manner (in the old time) did the holy women, which trusted in God, apparel themselves, being subject to their own husbands: as Sara obeyed Abraham calling him lord, whose daughters ye are made, doing well, and being not dismayed with any fear.
\begin{rubric}
	The newly married persons (the same day of their marriage) shall receive Holy Communion. 
\end{rubric}