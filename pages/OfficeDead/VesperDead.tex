\fancyhead[C]{{\LARGE Office of the Dead}}
\fancyhead[RO,LE]{\textit{Vespers}}
\section{Vespers of the Dead}
\begin{secrubric}
Vespers begins with the Lord's Prayer and Angelic Salutation, unless it should follow the carrying of the body to the Church or Matins \& Lauds of the occurrent Office, in which case the Office begins with the Antiphon.
\end{secrubric}

%MANUAL ADJUSTMENT:
\vspace{-0.25\baselineskip}

\subsection{Psalm 116. \textit{Dilexi, quoniam}}\par\noindent
\textit{Ant.} I will walk {\dag} before the Lord in the land of the living.\par
\lett{I}{am} well pleased : that the Lord hath heard the voice of my prayer;\par
\secondline{\psanum{2}That he hath inclined his ear unto me : therefore will I call upon him as long as I live.}
\thirdline{\psanum{3}The snares of death compassed me round about : and the pains of hell gat hold upon me.}
\psanum{4}I shall find trouble and heaviness, and I will call upon the Name of the Lord : O Lord, I beseech thee, deliver my soul.\par
\psanum{5}Gracious is the Lord, and righteous : yea, our God is merciful.\par
\psanum{6}The Lord preserveth the simple : I was in misery, and he helped me.\par
\psanum{7}Turn again then unto thy rest, O my soul : for the Lord hath rewarded thee.\par
\psanum{8}And why? thou hast delivered my soul from death : mine eyes from tears, and my feet from falling.\par
\psanum{9}I will walk before the Lord : in the land of the living.
\par
℣. Rest eternal * grant unto them, O Lord.\par
℟. And let light perpetual * shine upon them.\par\noindent
\textit{Ant.} I will walk before the Lord in the land of the living.
\subsection{Psalm 120. \textit{Ad Dominum}}\par\noindent
\textit{Ant.} Woe is me {\dag} that I am constrained to dwell with Mesech.\par
\lett{W}{hen} I was in trouble I called upon the Lord : and he heard me.\par
\secondline{\psanum{2}Deliver my soul, O Lord, from lying lips : and from a deceitful tongue.}
\thirdline{\psanum{3}What reward shall be given or done unto thee, thou false tongue : even mighty and sharp arrows, with hot burning coals.}
\psanum{4}Woe is me, that I am constrained to dwell with Mesech : and to have my habitation among the tents of Kedar.\par
\psanum{5}My soul hath long dwelt among them : that are enemies unto peace.\par
\psanum{6}I labour for peace, but when I speak unto them thereof : they make them ready to battle.\par
℣. Rest eternal * grant unto them, O Lord.\par
℟. And let light perpetual * shine upon them.\par\noindent
\textit{Ant.} Woe is me that I am constrained to dwell with Mesech.
\subsection{Psalm 121. \textit{Levavi oculus}}\noindent
\textit{Ant.} The Lord {\dag} shall preserve thee from all evil yea, it is even he that shall keep thy soul.
\lett{I}{will} lift up mine eyes unto the hills : from whence cometh my help.\par
\secondline{\psanum{2}My help cometh even from the Lord : who hath made heaven and earth.}
\thirdline{\psanum{3}He will not suffer thy foot to be moved : and he that keepeth thee will not sleep.}
\psanum{4}Behold, he that keepeth Israel : shall neither slumber nor sleep.\par
\psanum{5}The Lord himself is thy keeper : the Lord is thy defence upon thy right hand;\par
\psanum{6}So that the sun shall not burn thee by day : neither the moon by night.\par
\psanum{7}The Lord shall preserve thee from all evil : yea, it is even he that shall keep thy soul.\par
\psanum{8}The Lord shall preserve thy going out, and thy coming in : from this time forth for evermore.
\par
℣. Rest eternal * grant unto them, O Lord.\par
℟. And let light perpetual * shine upon them.\par\noindent
\textit{Ant.} The Lord shall preserve thee from all evil yea, it is even he that shall keep thy soul.

%MANUAL ADJUSTMENT:
\vspace{-1ex}
\subsection{Psalm 130. \textit{De profundis}}\noindent
\textit{Ant.} If thou, Lord, wilt be extreme {\dag} to mark what is done amiss, O Lord, who may abide it?\par
\lett{O}{ut} of the deep have I called unto thee, O Lord : Lord, hear my voice.\par
\secondline{\psanum{2}O let thine ears consider well : the voice of my complaint.}
\thirdline{\psanum{3}If thou, Lord, wilt be extreme to mark what is done amiss : O Lord, who may abide it?}
\psanum{4}For there is mercy with thee : therefore shalt thou be feared.\par
\psanum{5}I look for the Lord; my soul doth wait for him : in his word is my trust.\par
\psanum{6}My soul fleeth unto the Lord : before the morning watch, I say, before the morning watch.\par
\psanum{7}O Israel, trust in the Lord, for with the Lord there is mercy : and with him is plenteous redemption.\par
\psanum{8}And he shall redeem Israel : from all his sins.\par
℣. Rest eternal * grant unto them, O Lord.\par
℟. And let light perpetual * shine upon them.\par\noindent
\textit{Ant.} If thou, Lord, wilt be extreme to mark what is done amiss, O Lord, who may abide it?
\subsection{Psalm 138. \textit{Confitebor tibi}}\noindent
\textit{Ant.} Despise not then, {\dag} O Lord, the works of thine own hands.\par
\lett{I}{will} give thanks unto thee, O Lord, with my whole heart : even before the gods will I sing praise unto thee.\par
\secondline{\psanum{2}I will worship toward thy holy temple, and praise thy Name, because of thy loving-kindness and truth : for thou hast magnified thy Name and thy word above all things.}
\thirdline{\psanum{3}When I called upon thee, thou heardest me : and enduedst my soul with much strength.}
\psanum{4}All the kings of the earth shall praise thee, O Lord : for they have heard the words of thy mouth.\par
\psanum{5}Yea, they shall sing in the ways of the Lord : that great is the glory of the Lord.\par
\psanum{6}For though the Lord be high, yet hath he respect unto the lowly : as for the proud, he beholdeth them afar off.\par
\psanum{7}Though I walk in the midst of trouble, yet shalt thou refresh me : thou shalt stretch forth thy hand upon the furiousness of mine enemies, and thy right hand shall save me.\par
\psanum{8}The Lord shall make good his loving-kindness toward me : yea, thy mercy, O Lord, endureth for ever; despise not then the works of thine own hands.
\par
℣. Rest eternal * grant unto them, O Lord.\par
℟. And let light perpetual * shine upon them.\par\noindent
\textit{Ant.} Despise not then, O Lord, the works of thine own hands.\par
\vspace{0.5\baselineskip}
℣. I heard a voice from heaven, saying unto me.\par 
℟. Blessed are the dead which die in the Lord.\par
%\elcol{℣. I heard a voice from heaven, saying unto me.}{℣. Audívi vocem de c{\ae}lo dicéntem mihi.}
%\elcol{℟. Blessed are the dead which die in the Lord.}{℟. Beáti mórtui qui in Dómino moriúntur.}
\subsection{Magnificat}
\noindent
\textit{Ant.} All that the Father {\dag} giveth me shall come to me; and him that cometh to me, I will in no wise cast out.
\par
\lett{M}{y} soul {\ding{64}} doth magnify the Lord, * and my spirit hath rejoiced in God my Saviour.\par
\secondline{For he hath regarded * the lowliness of his handmaiden.}
\thirdline{For behold, from henceforth * all generations shall call me blessed.}
    For he that is mighty hath magnified me; * and holy is his Name.\par
    And his mercy is on them that fear him * throughout all generations.\par
    He hath showed strength with his arm; * he hath scattered the proud in the imagination of their hearts.\par
    He hath put down the mighty from their seat, * and hath exalted the humble and meek.\par
    He hath filled the hungry with good things; * and the rich he hath sent empty away.\par
    He remembering his mercy hath holpen his servant Israel; * as he promised to our forefathers, Abraham and his seed, for ever.

℣. Rest eternal * grant unto them, O Lord.

℟. And let light perpetual * shine upon them.
\par\noindent
	\textit{Ant.} All that the Father giveth me shall come to me; and him that cometh to me, I will in no wise cast out.

\subsection{Lord's Prayer}
\begin{rubric}
    {The Lord's Prayer is here said, in secret, kneeling, ending with,}
\end{rubric}
℣. And lead us not into temptation.

℟. But deliver us from evil.

\subsection{Psalm 146. \textit{Lauda, anima mea}}
\begin{rubric}
{Psalm 146 is not said on All Souls Day, on the day of death or burial, nor at any time when the Office is recited with Double rite.}
\end{rubric}
\lett{P}{raise} the Lord, O my soul; while I live will I praise the Lord : yea, as long as I have any being, I will sing praises unto my God.\par
\secondline{\psanum{2}O put not your trust in princes, nor in any child of man : for there is no help in them.}
\thirdline{\psanum{3}For when the breath of man goeth forth he shall turn again to his earth : and then all his thoughts perish.}
\psanum{4}Blessed is he that hath the God of Jacob for his help : and whose hope is in the Lord his God;\par
\psanum{5}Who made heaven and earth, the sea, and all that therein is : who keepeth his promise for ever;\par
\psanum{6}Who helpeth them to right that suffer wrong : who feedeth the hungry.\par
\psanum{7}The Lord looseth men out of prison : the Lord giveth sight to the blind.\par
\psanum{8}The Lord helpeth them that are fallen : the Lord careth for the righteous.\par
\psanum{9}The Lord careth for the strangers, he defendeth the fatherless and widow : as for the way of the ungodly, he turneth it upside down.\par
\psanum{10}The Lord thy God, O Sion, shall be King for evermore : and throughout all generations.
\par
℣. Rest eternal * grant unto them, O Lord.\par
℟. And let light perpetual * shine upon them.

\subsection{Responsory}
℣. From the gate of hell.

℟. Deliver \textit{his soul}, O Lord.

℣. May \textit{he} rest in peace.

℟. Amen.

\subsection{Conclusion}
℣. O Lord, hear my prayer.

℟. And let my cry come unto thee.

℣. The Lord be with you.

℟. And with thy spirit.

℣. Let us pray.

\begin{rubric}
    {The appropriate collect is here said (p. \pageref{dead}).}
\end{rubric}
℣. Rest eternal * grant unto them, O Lord.

℟. And let light perpetual * shine upon them.

℣. May they rest in peace.

℟. Amen.

\begin{center}
    \textsc{Here endeth the Order of Vespers of the Dead.}
\end{center}